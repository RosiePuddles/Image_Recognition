\documentclass[twocolumn,twoside]{article}
\usepackage[utf8]{inputenc}
\usepackage[english]{babel}

\usepackage{hyperref}
\usepackage{fancyhdr}
\usepackage{pgfplots}
\usepackage{graphicx}
\usepackage{indentfirst}
\usepackage{mathtools,amssymb}
\usepackage{chngcntr}

\numberwithin{equation}{section}
\counterwithin{figure}{section}

\pagestyle{fancy}
\fancyhf{}
\fancyhead[LO,RE]{\nouppercase{\leftmark}}
\fancyhead[RO,LE]{Detection of a non-linearly scaled sub-image in a parent image}
\cfoot[\thepage]{\thepage}

\title{Detection of a scaled and resized sub-image in a parent image}
\date{January 2021}
\author{
  Bartlett, Rosie\\
  \texttt{rosiegbartlett@gmail.com}
}

\begin{document}
\maketitle
\begin{abstract}
This paper is an outline of several proposed methods to detect a non-linearly scaled sub-image inside a parent image. I will discuss the different aspects of each method, comparing the efficiency and accuracy of each method for differently sized images.
\end{abstract}
\section{Proposed method}
\subsection{Fourier Series Approximation}

The proposed method to detect a sub-image inside its parent image uses fast Fourier transforms to approximate each row of the image as a Fourier series, and then compare the original values of the approximation at individual points, along with several derivatives of the appproximation. This ammounts to checking that the colour of the images at the chosen points is the same, and that the context of the points are the same. For the points used, the top left corner of the sub-image will have it's $C$ matrix calulated, which I will refer to as $C_0$, which will be used a benchmark for all other points until a match is found. The $C$ matrix refers to \autoref{eq:C}, where $f_n(t)$ is the Fourier approximation of the array of values representing the $n$\textsuperscript{th} row of the image, and $m$ is the upper limit of derivatives chosen; which will be explored further.
\begin{equation}\label{eq:C}
  C=
  \begin{bmatrix}
    f_n(t)\\{f_n}'(t)\\{f_n}''(t)\\\vdots\\{f_n}^m(t)
  \end{bmatrix}
\end{equation}

Once $C_0$ has been calulated, the code will work from the top left corner of the parent image in the following manner:
\begin{enumerate}
  \item Calulate the Fourier coefficients matrix%\textsuperscript{\cite{StackOverflow_fourier_coeffs}}
  , $F$, for the current row of pixels
  \item Calulate the coefficients of the derivatives of the approximation, upto and including the $m$\textsuperscript{th} derivative
  \item For each pixel in the current parent image row, calulating the $C$ matrix for each, using the appropriate value of $t$
  \item The current $C$ is then compared to $C_0$, and if the requirements are met then the code assumes a point to have been found on the parent image that correlates with the upper left point in the sub-image. These requirements are detailed below.
\end{enumerate}

For the code to assume a point to have been found to correlate with the upper left corner of the sub-image, it first checks divides $C_0$ by $C$ elementwise, resulting in $R_0$, which is saved. It then calulates the mean of the elements of $R_0$, giving $\bar{R_0}$. These are then used to calulate the mean squared error of the matrix, as described in \autoref{eq:mse}
\begin{equation}\label{eq:mse}
  \frac{1}{m+1}\sum_{n=0}^m{\left(R_0[n]-\bar{R_0}\right)^2}
\end{equation}

If the result of this is less than some constant $d$, the threshold error, then the requirements are said to have been met and this the top left corner of the sub-image is assumed to have been found. For further notation, this pixel will be refered to as $p_0$.\\

Once this point has been found, the code then goes back to the sub-image to calulate the $C$ matrix for the top right pixel, $C_1$. This is then compared with pixels to the right of $p_0$. If a matching pixel is found in the method described above, replacing $C_0$ and $R_0$ with $C_1$ and $R_1$ respectively, a further check is required. The value of $R_0$ is checked aginst $R_1$ to ensure that they are similar, by calulating the mean squared error between the two matrices and checking to make sure this is below a specifiec threshold value. If this extra condition is met, then the point at which these conditions are met is assumed to be the top right corner of the sub-image; this point is $p_1$.\\

The search for $p_1$ is only performed along the same row as the row $p_0$ is on since the image has not been rotated. If $p_1$ is not found on the same row as $p_0$, then $p_0$ is disregarded and the code begins the search for $p_0$ from the previous position of $p_0$ until it is found again.\\

Once both $p_0$ and $p_1$ are located, the code must then locate $p_2$ and $p_3$, the bottom left and right pixels of the image respectively. To do this, the parent image is rotated 90º and the Fourier coefficients are calulated for the new rows that $p_0$ and $p_1$ lie on. The code then goes along the row $p_0$ is on until the coditions for $p_1$ are met for a point on the row. The code then checks the position on the row with $p_1$ on with the same offset as the located $p_2$ to check that the value is as expected. If it is then the poition of the sub-image is assumed to have been found and is returned. If this is not met, then the code disregards the location of $p_2$ and continues searching for an appropriate $p_2$ for which the coditions for $p_3$ are met.

\end{document}
